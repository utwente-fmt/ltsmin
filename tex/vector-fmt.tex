\documentclass{article}
\title{The vector file format for labeled transition systems.}
\author{Stefan Blom}

\usepackage{tabularx}
\usepackage{fullpage}

%% Set up theorems and such...

\usepackage{theorem}
\theoremstyle{plain}
\theoremheaderfont{\bfseries\upshape}
\theorembodyfont{\mdseries\upshape}

\newtheorem{theorem}{Theorem}[section]
\newtheorem{intermezzo}[theorem]{Intermezzo}
\newtheorem{example}[theorem]{Example}
\newtheorem{lemma}[theorem]{Lemma}
\newtheorem{corollary}[theorem]{Corollary}
\newtheorem{conjecture}[theorem]{Conjecture}
\newtheorem{definition}[theorem]{Definition}
\newtheorem{proposition}[theorem]{Proposition}
\newtheorem{fact}[theorem]{Fact}
\newtheorem{remark}[theorem]{Remark}
\newenvironment{proof}[1][Proof.~]{{\bf #1}}{\mbox{}\hfill$\Box$\medskip}

%% End theorem setup.

\begin{document}

\maketitle

\section{Introduction.}

One of the key purposes of the file format is the ability to support writing generic tools.
Both integer types and arbitrary types are internally represented as integers.
For the arbitrary types a serialization of the values is included. Using a human reqadable form is recommended because this enables generic tools to provide meaningful feedback.

To support features like composition and quantification is generic tools, it is important to know
if a type is finite or not. Thus we get four possibilities:

\centerline{\begin{tabular}{lll}
& integer & aribitrary
\\
finite & $[.,.]$ & enum
\\
possibly infinite & direct & chunk
\end{tabular}}


\section{Vector Format Files}

The files, which are possible in the vector file format are:

\par\noindent\begin{tabularx}{\textwidth}{lX}
{\bf file} & {\bf description}
\\
\verb+info+ & Contains the meta information of the LTS.
\\
\verb+init+ & Contains the initial states of the LTS.
\\
\verb+SV-+$i$\verb+-+$k$ & Contains the State Vector values of slot $k$ for states in partition $i$.
\\
\verb+SL-+$i$\verb+-+$k$ & Contains the State Label values of label $k$ for states in partition $i$.
State labels, which are also state slots are stored under the slot number.
\\
\verb+ES-+$i$\verb+-+$k$ & Contains the Edge Source values of slot $k$ for edges in partition $i$.
\\
\verb+ES-+$i$\verb+-seg+ & Contains the Edge Source segment for edges in partition $i$.
\\
\verb+ES-+$i$\verb+-ofs+ & Contains the Edge Source state number for edges in partition $i$.
\\
\verb+ED-+$i$\verb+-+$k$ & Contains the Edge Destination value of slot $k$ for edges in partition $i$.
\\
\verb+ED-+$i$\verb+-seg+ & Contains the Edge Destination segment for transitions in partition $i$.
\\
\verb+ED-+$i$\verb+-ofs+ & Contains the Edge Destination state number for transitions in partition $i$.
\\
\verb+EL-+$i$\verb+-+$k$ & Contain the Edge Label value of label $k$ for the edges in partition $i$.
\\
\verb+CT-+$k$ & Stores serialized values of the sort $k$.
\end{tabularx}


The format of the info file is:

\par\noindent\begin{tabularx}{\textwidth}{lX}
{\tt string} & A string describing the file format and version: "vector 1.0".
\\
{\tt string} & Comment string.
\\
{\tt chunk} & Serialization of the LTS type.
\\
{\tt chunk} & Contains the structural information of the file.
\\
{\tt chunk} & The state and transition counts of each partition, the number of elements in each chunk table.
\\
{\tt chunk} & A serialization of the compression tree. If the chunk size is 0 then
tree compression was not used.
\end{tabularx}

The format for the structural information chunk is:
\par\noindent\begin{tabularx}{\textwidth}{lX}
{\tt int32} & The number of segments.
\\
{\tt string} & The representation if the initial states: "i", "v" , or "sv".
Note that "i" means one integer (state number) if the segment count is 1 and two integers
(segment,offset) if the segment count is more than 1.
\\
{\tt string} & the edge ownership: "src" or "dst".
\\
{\tt string} & The representation of the source states: "i", "v", or "sv".
\\
{\tt string} & The representation of the destination states: "i", "v", or "sv".
\end{tabularx}

\section{Compression and Archives}

Managing a set of files is slightly more difficult and error prone than managing a single file.
Unfortunately, existing archive file formats such as ZIP and tar cannot be used easily
because multiple streams have to be written in parallel. This lead to the development of
our own archive format: the Generic Container Format (GCF).

An additional advantage of using our own format is that we can support various compression
techniques:
\\
\begin{tabularx}{\textwidth}{lX}
{\tt gzip[N]} & Apply gzip compression. The level $N$ is optional.
\\
{\tt diff32} & Assume that the stream contains 32bit integer and apply difference encoding.
This is known to be useful for the source indices of transitions, which are often ascending sequences.
The default strategy for compressing state indices is {\tt diff32|gzip}.
\\
{\tt rle32} & A version of run-length encoding that support runs of identical 32 bit integer
and runs of always changing 32 bit integer. It adds a small overhead to random files,
but compresses files with long runs of the same value very well.
The default strategy for compressing state labels is {\tt rle32|gzip} and is very effective
for Markov chains in TRA/LAB form.
\end{tabularx}

\section{Serialization of the LTS type.}

A version 1.0 LTS type is serialized as follows:

\par\noindent\begin{tabularx}{\textwidth}{lX}
{\tt string} & A string describing the version: "lts signature 1.0".
\\
{\tt int32} & The length of the state vector: vlen.
\\
({\tt string} {\tt int32})${}^{\rm vlen}$ & For each state vector slot a string describing the name and an integer that describes the type.
An empty string means undefined.
\\
{\tt int32} & The number of defined state labels: sl.
\\
({\tt string} {\tt int32})${}^{\rm sl}$ & For each defined state label a string describing the name and an integer that describes the type.
\\
{\tt int32} & The number of edge labels: el.
\\
({\tt string} {\tt int32})${}^{\rm el}$ & For each edge label a string describing the name and an integer that describes the type.
\\
{\tt int32} & The number of types: T.
\\
{\tt string}${}^{\rm T}$ & For each type a string describing the name of the type.
\end{tabularx}

It is assumed that every type uses a chunk representation.

\bigskip

A version 1.1 LTS type is serialized as follows:

\par\noindent\begin{tabularx}{\textwidth}{lX}
{\tt string} & A string describing the version: "lts signature 1.1".
\\
{\tt int32} & The length of the state vector: vlen.
\\
({\tt string} {\tt int32})${}^{\rm vlen}$ & For each state vector slot a string describing the name and an integer that describes the type.
An empty string means undefined.
\\
{\tt int32} & The number of defined state labels: sl.
\\
({\tt string} {\tt int32})${}^{\rm sl}$ & For each defined state label a string describing the name and an integer that describes the type.
\\
{\tt int32} & The number of edge labels: el.
\\
({\tt string} {\tt int32})${}^{\rm el}$ & For each edge label a string describing the name and an integer that describes the type.
\\
{\tt int32} & The number of types: T.
\\
({\tt string} {\tt string})${}^{\rm T}$ & For each type a string describing the name of the type and a string describing how values are represented.
\end{tabularx}

Legal values representations are:
\begin{description}
\item[direct] Arbitrary 32 bit integer value.
\item[\mbox{$[n,k]$}] Integer in the range $[n,k]$.
\item[chunk] Every element is a chunk, from an unknown domain.
\item[enum] Every element is a chunk, the entire domain is given.
\end{description}


\section{Serialization of basic data types.}

\par\noindent\begin{tabularx}{\textwidth}{@{}lX@{}}
Datatype & description
\\
{\tt int16} & 16 bit integer in network byte order.
\\
{\tt int32} & 32 bit integer in network byte order.
\\
{\tt int64} & 64 bit integer in network byte order.
\\
{\tt string} & Byte strings of less than $2^{16}$ bytes. A byte string of length $N$ is encoded as
$N$ in {\tt int16} format followed by the $N$ bytes of the string. When used for printable strings the terminating $0$ is not included.
\\
{\tt length} & An unsigned integer with a variable length encoding.
A length is encoded as a sequence of bytes. The last byte in the sequence has bit 7 set to 0, all other bytes have bit 7 set to 1. The byte order is little endian. For example:
\\&
\begin{tabular}{rr}
$N$ & encoding
\\
16 & 0001\,0000
\\
256 & 1000\,0001~0000\,0000
\end{tabular}
\\
{\tt chunk} & Byte strings of arbitrary length, represented as a {\tt length} followed by the bytes of the string.
\end{tabularx}

\end{document}

