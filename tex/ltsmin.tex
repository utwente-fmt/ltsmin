\documentclass{report}

\title{The LTSmin toolset}
\author{Stefan Blom, Jaco van de Pol and Michael Weber}

\usepackage{tabularx}
\usepackage[all]{xypic}

\newcommand{\struct}[1]{\ensuremath{\langle#1\rangle}}

%% Set up theorems and such...

\usepackage{theorem}
\theoremstyle{plain}
\theoremheaderfont{\bfseries\upshape}
\theorembodyfont{\mdseries\upshape}

\newtheorem{theorem}{Theorem}[section]
\newtheorem{intermezzo}[theorem]{Intermezzo}
\newtheorem{example}[theorem]{Example}
\newtheorem{lemma}[theorem]{Lemma}
\newtheorem{corollary}[theorem]{Corollary}
\newtheorem{conjecture}[theorem]{Conjecture}
\newtheorem{definition}[theorem]{Definition}
\newtheorem{proposition}[theorem]{Proposition}
\newtheorem{fact}[theorem]{Fact}
\newtheorem{remark}[theorem]{Remark}
\newenvironment{proof}[1][Proof.~]{{\bf #1}}{\mbox{}\hfill$\Box$\medskip}

%% End theorem setup.

\begin{document}

\maketitle

\tableofcontents


\chapter{Design Rationale}

\section{Management of data values.}

All data type functionality is provided by the languages modules.
The tool level itself can only manage integers, chunks and strings.
Thus, every other data type has to be serialized to a chunk
and/or pretty printed to a string.

While this approach makes the tool design simpler, it does add cost
in time and memory: (de)serialization is expensive in time and the
serialized form often takes up a lot more memory than the real data
type. For example, it takes time to print/parse ATerms. Moreover,
ATerm use maximal sharing to reduce the memory footprint which is not
possible in the serialized form.

Because of the time cost, we try to avoid (de)serialization by
keeping a global integer to value mapping, so we can send integers
instead of chunks. Because of the memory cost, duplication of
serialized values should be avoided. That is, tools that start language
modules should use the language modules to store values in native form
and the file format should store all values for each type once.

This requirement is somewhat at odds with the fact that the performance
of the symbolic data structures depends on the integers used.
A variable with integers $1,2,3,4$ performs better than one with
$1,5,45,81$, simply because the first set can be represented with less bits.
This is solved by introducing a global map for each variable that maps
the used indices to a consecutive range.

\chapter{The Enumerated Table Format}


The problem with a model given by means of a partitioned
next state function is that one needs to do a reachability analysis to
compute the transitions relations and state label mappings.
Hence it is convenient to store the computed transition relations
and state label mappings in a file that can itself be
used as an input to other tools. Because we do this by enumerating
transitions and mappings, rather than by writing  some
form of decision diagram, we refer to the resulting intermediate
format as Enumerated Table Format (ETF).

\section{Introduction}

An ETF file is a simple text file that describes a grey box model,
it consists of several sections. Each ETF file contains the following sections:
\begin{itemize}
\item A single {\em state} section. This must be the first section in the file.
It describes the structure of the state vector.
\item A single {\em edge} section. This must be the second section in the file.
It describes the names and types of the edge labels.
\item A single {\em init} section. This section describes the set of initial states.
Please note that tool support assumes that this set is singleton.
\item Zero or more {\em trans} sections. Each section describes a set of transitions
in the model and the attached edge labels. Semantically a single section would suffice,
but having one section for each edge group allows a simpler implementation.
\item Zero or more {\em map} sections. Each section defines the value of a state variable.
\item Zero or more {\em sort} sections. Each section enumerates the values of one sort.
\end{itemize}

As a running example, we will use the LTS below.
\[
{\xymatrix@=4em{
*+[F=]{\,0\,0\,} \ar@{->}[r]^a\ar@{->}[d]_b
&
*+[F]{\,0\,1\,} \ar@{->}[d]^b
\\
*+[Fo]{\,1\,0\,} \ar@{->}[r]_a
&
*+[Fo]+=+[Fo]{\,1\,1\,}
}}
\]
That is the state is a pair of bits, the single edge label is $a$ or $b$ and there are
two state labels: shape (circle or rectangle) and multiplicity (single or double).

\section{The state section.}

The state section of our example is:
\begin{verbatim}
begin state
x:bit y:bit
end state
\end{verbatim}
Which means that each state has length 2, the first element is called x and of type bit,
the second element is called y and of type bit as well. For this LTS, we do not need to know the
names of the variables, so we can replace those with the special symbol \verb+_+  which denotes unknown.
\begin{verbatim}
begin state
_:bit _:bit
end state
\end{verbatim}
We also do not need to know the type of the variables, so we can omit those as well:
\begin{verbatim}
begin state
_:_  _:_
end state
\end{verbatim}

\section{The edge section}

The edge section is similar in form to the state section. The only difference is that
names and types for edge labels is mandatory, so the only legal form is:
\begin{verbatim}
begin edge
action:action
end edge
\end{verbatim}

\section{The init section}

Assuming that the initial state is $0\,0$, the init section of our example is:
\begin{verbatim}
begin init
0 0
end init
\end{verbatim}

While the syntax allows for an arbitrary number of initial states.
We require the use of precisely one initial state.

\section{The trans sections}

The transitions of the LTS can be described in many ways.
The simplest is:

\begin{verbatim}
begin trans
0/1 * a
* 0/1 b
end trans
\end{verbatim}
That is an $a$ actions changes the first element from 0 to 1 and
a $b$ action changes the second element from 0 to 1. However this section lacks
an important property. A section is called uniform if for every column it
holds that either every element is a \verb+*+ or none of the elements is \verb+*+.
Tools are allowed to require uniform input. A uniform representation
of the transition relation is:
\begin{verbatim}
begin trans
0/1 * a
end trans
begin trans
* 0/1 b
end trans
\end{verbatim}
If the labels get big then we want to use table compression on them.
That is supported by using numbers instead of string. E.g.
\begin{verbatim}
begin trans
0/1 * 0
end trans
begin trans
* 0/1 1
end trans
\end{verbatim}
We will see later how to specify the values of 0 and 1.
If you want to specify the string 0 then you need to write \verb+"0"+.

\section{The map sections}

Our two maps are specified as follows:
\begin{verbatim}
begin map shape:shape
0 * square
1 * circle
end map
begin map multiplicity:multiplicity
0 0 double
0 1 single
1 0 single
1 1 double
end map
\end{verbatim}
Note how the map multiplicity enumerates every reachable state explicitly.
This is because the multiplicity depends on the parity of the state vector.
In general, every map that depends on the entire state vector can leads
to an extremely big table. 

\section{The sort sections}

Earlier we used references to the action sort:
\begin{verbatim}
begin trans
0/1 * 0
end trans
begin trans
* 0/1 1
end trans
\end{verbatim}
To make that precise, we need to add the section

\begin{verbatim}
begin sort action
a
b
end sort
\end{verbatim}

\section{Summary}

So one possible ETF specification of our LTS is

\begin{verbatim}
begin state
_:_  _:_
end state
begin edge
action:action
end edge
begin init
0 0
end init
begin trans
0/1 * 0
end trans
begin trans
* 0/1 1
end trans
begin map shape:shape
0 * square
1 * circle
end map
begin map multiplicity:multiplicity
0 0 1
0 1 0
1 0 0
1 1 1
end map
begin sort action
a
b
end sort
begin sort multiplicity
single
double
end sort
\end{verbatim}




\chapter{File Formats}

\section{DIRectory format}

\section{Vector Format}

A vector format file, stores the states and transitions
in $W$ partitions each. For each partition, a list of source states and state labels is stored
as one file per state slot and state label. This defines a segmented state numbering.
The transitions are stored as one file containing the state number of a source state
plus one file per destination state slot and edge label.

\par\noindent\begin{tabularx}{\textwidth}{lX}
\verb+info+ & Stores all information that is needed but does not need a file of its own.
\\
\verb+SV-+$i$\verb+-+$k$ & Contains the State Vector slot value for segment $i$ and slots $k$.
If $N = 1$ and $S^i$ is of the form $\{ 0 , \cdots , |S^i| -1 \}$ then the
file \verb+SV-1-1+ can be omitted because it is a simple sequence $0,1,\cdots,|S^1|-1$.
\\
\verb+SL-+$i$\verb+-+{\it name} & Contains the value of state label {\it name} for the states in
segment $i$.
\\
\verb+ES-+$i$ & Contains the Edge Source state number for transitions starting in segment $i$.
\\
\verb+ED-+$i$\verb+-+$k$ & Contains the Edge Destination state slot value for 
segment $i$ and slot $k$.
\\
\verb+EL-+$i$\verb+-+{\it name} &  Contain the Edge Label value for the edge label {\it name} for
edges starting in segment $i$.
\\
\verb+CT-+{\it name} & Stores serialized values of the sort {\it name}.
\\
\verb+PP-+{\it name} & Stores pretty printed values of the sort {\it name}.
\end{tabularx}


\end{document}

